\documentclass[12pt,a4paper]{report}
\usepackage{graphicx}
\usepackage{booktabs}
\usepackage{geometry}
\usepackage{titlesec}
\usepackage{hyperref}
\usepackage{amsmath}
\usepackage{array}
\usepackage{fancyhdr}
\usepackage{xcolor}

\geometry{margin=1in}
\setlength{\parskip}{6pt}
\setlength{\parindent}{0pt}

\titleformat{\chapter}[hang]{\bfseries\Huge}{\thechapter.}{1em}{}
\titleformat{\section}[hang]{\bfseries\Large}{\thesection}{1em}{}

\pagestyle{fancy}
\fancyhf{}
\rhead{\thepage}
\lhead{Sorting Algorithms - Comparative Study}

\begin{document}

\begin{titlepage}
    \centering
    \vspace*{3cm}
    {\Huge\bfseries EXPERIMENTAL COMPLEXITY ANALYSIS\par}
    \vspace{0.5cm}
    {\Huge\bfseries SORTING ALGORITHMS – COMPARATIVE STUDY\par}
    \vspace{1cm}
    {\Large Design and Analysis of Algorithms (DAA) Project\par}
    \vspace{2cm}
    {\large \textbf{Submitted by:} Rohan \\[0.5cm]
    Department of Computer Science and Engineering \\[0.2cm]
    SRM University, AP\par}
    \vfill
    \vspace{0.5cm}
    \vspace{0.5cm}
    \vspace*{1cm}
    {\large \textbf{Date:} \today}
    \vspace*{1cm}
\end{titlepage}


\tableofcontents
\newpage

\chapter{Introduction}
Sorting is a fundamental operation in computer science, forming the basis for efficient searching, data analysis, and numerous algorithms. This project conducts a detailed experimental analysis of classical and modern sorting algorithms, comparing their correctness, time complexity, and practical performance under different input conditions.

\section{Objectives}
\begin{itemize}
    \item To implement and verify correctness of ten sorting algorithms.
    \item To analyze performance under best, average, and worst case scenarios.
    \item To experimentally verify theoretical time complexities.
    \item To identify the most efficient algorithms for different data scales.
\end{itemize}

\chapter{Algorithms Studied}
\begin{enumerate}
    \item Bubble Sort
    \item Selection Sort
    \item Insertion Sort
    \item Merge Sort
    \item Quick Sort (Deterministic)
    \item Quick Sort (Randomized)
    \item Heap Sort
    \item Counting Sort
    \item Radix Sort
    \item Bucket Sort
\end{enumerate}

\section{Complexity Overview}
\begin{center}
\begin{tabular}{|l|c|c|c|}
\hline
\textbf{Algorithm} & \textbf{Best Case} & \textbf{Average Case} & \textbf{Worst Case} \\
\hline
Bubble Sort & $O(n)$ & $O(n^2)$ & $O(n^2)$ \\
Selection Sort & $O(n^2)$ & $O(n^2)$ & $O(n^2)$ \\
Insertion Sort & $O(n)$ & $O(n^2)$ & $O(n^2)$ \\
Merge Sort & $O(n \log n)$ & $O(n \log n)$ & $O(n \log n)$ \\
Quick Sort (Det) & $O(n \log n)$ & $O(n \log n)$ & $O(n^2)$ \\
Quick Sort (Rand) & $O(n \log n)$ & $O(n \log n)$ & $O(n \log n)$ \\
Heap Sort & $O(n \log n)$ & $O(n \log n)$ & $O(n \log n)$ \\
Counting Sort & $O(n + k)$ & $O(n + k)$ & $O(n + k)$ \\
Radix Sort & $O(nk)$ & $O(nk)$ & $O(nk)$ \\
Bucket Sort & $O(n + k)$ & $O(n + k)$ & $O(n^2)$ \\
\hline
\end{tabular}
\end{center}

\chapter{Experimental Setup}
\section{System Configuration}
\begin{itemize}
    \item OS: Arch Linux
    \item CPU: Intel Core i5 (13th Gen)
    \item RAM: 16 GB
    \item Language: Python 3
    \item Timer: \texttt{time.perf\_counter()} for high-resolution timing
\end{itemize}

\section{Test Conditions}
Each algorithm was tested under:
\begin{itemize}
    \item Random data (average case)
    \item Sorted data (best case)
    \item Reverse sorted data (worst case)
\end{itemize}

Input sizes used: \\
$[100, 500, 1000, 5000, 10000, 50000, 100000, 500000, 1000000, 5000000, 10000000]$

\chapter{Results and Analysis}
\section{Correctness Testing}
All algorithms passed the correctness tests on various datasets.

\section{Performance Analysis}
The following trends were observed:

\subsection*{Complexity Verification}
\begin{itemize}
    \item $O(n^2)$ algorithms (Bubble, Selection, Insertion) showed exponential slowdown with increasing $n$.
    \item $O(n \log n)$ algorithms (Merge, Quick-Rand, Heap) scaled predictably.
    \item Linear algorithms (Counting, Radix, Bucket) exhibited near-linear scaling.
\end{itemize}

\subsection*{Best vs Worst Case}
\begin{itemize}
    \item Bubble Sort: $\sim$30,000x faster on sorted vs random data.
    \item Insertion Sort: $\sim$10,000x faster on sorted vs random data.
    \item Deterministic Quick Sort degraded to $O(n^2)$ for sorted/reverse data.
    \item Randomized Quick Sort maintained consistent performance.
\end{itemize}

\subsection*{Algorithm Rankings (n = 10,000,000)}
\textbf{Random Data (Average Case):}
\begin{enumerate}
    \item Counting Sort (~2.3s)
    \item Bucket Sort (~5.1s)
    \item Radix Sort (~11.2s)
    \item Quick Sort (Det) (~19.9s)
    \item Quick Sort (Rand) (~20.3s)
    \item Merge Sort (~24.6s)
    \item Heap Sort (~32.9s)
\end{enumerate}

\textbf{Sorted Data (Best Case):}
\begin{enumerate}
    \item Bubble Sort (~0.03s)
    \item Insertion Sort (~0.05s)
    \item Counting Sort (~2.2s)
    \item Bucket Sort (~4.2s)
\end{enumerate}

\textbf{Reverse Sorted Data (Worst Case):}
\begin{enumerate}
    \item Counting Sort (~2.3s)
    \item Bucket Sort (~4.5s)
    \item Radix Sort (~10.7s)
    \item Merge Sort (~13.2s)
\end{enumerate}

\begin{figure}
    \centering
    \includegraphics[width=1\linewidth]{all_algorithms_graph.png}
    \caption{All Sorting Graphs}
    \label{fig:placeholder}
\end{figure}

\chapter{Discussion}
\section{Observations}
\begin{itemize}
    \item Randomized Quick Sort is the most reliable general-purpose sorter.
    \item Merge Sort offers predictable $O(n \log n)$ performance but with higher memory usage.
    \item Counting, Bucket, and Radix Sort dominate large integer datasets due to linear time performance.
    \item Heap Sort is useful when memory constraints are critical.
\end{itemize}

\section{Space Complexity}
\begin{itemize}
    \item In-place: Bubble, Selection, Insertion, Quick Sort, Heap Sort
    \item Extra Space: Merge Sort, Counting Sort, Bucket Sort, Radix Sort
\end{itemize}

\chapter{Conclusion}
The experimental study confirms that theoretical complexities align well with practical results.  
For small datasets, insertion sort performs efficiently; for medium to large datasets, randomized quick sort and merge sort offer excellent performance; and for very large integer-based datasets, counting or radix sort are optimal choices.

\chapter*{References}
\begin{itemize}
    \item Cormen, T. H., et al. \textit{Introduction to Algorithms}, MIT Press.
    \item Sedgewick, R. \textit{Algorithms in C++}, Addison-Wesley.
    \item Levitin, A. \textit{Introduction to the Design and Analysis of Algorithms}, Pearson.
\end{itemize}

\end{document}
